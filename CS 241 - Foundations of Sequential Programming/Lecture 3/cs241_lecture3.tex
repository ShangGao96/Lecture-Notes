\documentclass{report}
\usepackage[margin=1in, paperwidth=8.5in, paperheight=11in]{geometry}
%Math packages%
\usepackage{amsmath}
\usepackage{amsthm}
%Spacing%
\usepackage{setspace}
\onehalfspacing
%Lecture number%
\newcommand{\lectureNum}{3}
%Variables - Date and Course%
\newcommand{\curDate}{January 10, 2017}
\newcommand{\course}{CS 241}
\newcommand{\instructor}{Kevin Lanctot}
%Defining the example tag%
%\theoremstyle{definition}%
\newtheorem{ex}{Example}[section]
%Setting counter given the lecture number%
\setcounter{chapter}{\lectureNum{}}
%Package to insert code%
\usepackage{listings}
\usepackage{courier}
\usepackage{xcolor}
\lstset { %
    tabsize=2,
    breaklines=true,
    language=C++,
    backgroundcolor=\color{blue!8}, % set backgroundcolor
    basicstyle=\footnotesize\ttfamily,% basic font setting
}
\begin{document}
%Note title%
\begin{center}
\begin{Large}
\textsc{\course{} | Lecture \lectureNum{}}
\end{Large}
\end{center} 
\noindent \textit{Bartosz Antczak} \hfill
\textit{Instructor: \instructor{}} \hfill
\textit{\curDate{}}
\rule{\textwidth}{0.4pt}
% Actual Notes%
\subsubsection{Review of last lecture}
Computers only understand binary (strings of ones and zeros). Humans program computers using high-level languages (such as Java, C++). To convert from high-level language to binary, we use an assembly language. This course will cover the MIPS assembly language. MIPS contains a set of instructions and registers that work with data. High-level code gets converted to these instructions. We've covered some of these instructions such as \texttt{add} or \texttt{jr}.
\subsection{What's in a Computer?}
\begin{itemize}
\item \textbf{RAM (Random Access Memory)}: stores data when the computer is on, it's essentially a massive array
\item \textbf{Processor} manipulates data; it consists of two parts
\begin{itemize}
\item \textit{control unit}: controls the flow of data throughout the processor, RAM, and registers
\item \textit{data path}: processes the data. The major components in it include:
\begin{itemize}
\item Program Counter (PC): holds the address of the current instruction
\item Instruction Register (IR): holds the instruction that is being executed
\item Arithmetic Logic Unit (ALU): performs arithmetic calculations
\item General Purpose Registers: temporary storage within the processor (i.e., \$4)
\end{itemize} 
\end{itemize}
\end{itemize}
\section{Conditional Execution in MIPS}
In C++, we have
\begin{lstlisting}
if ... else if ... else
while () { ... }
for () { ... }
\end{lstlisting}
But in MIPS, we have a different set of instructions for conditional statements and loops
\begin{itemize}
\item \texttt{beq}: branch if equal. The instruction is structured as \texttt{beq \$s, \$t, i}, which compares the contents of registers \$s and \$t. If they're equal, we skip \texttt{i} instructions. \texttt{i} can be positive or negative
\item \texttt{bne}: branch if not equal. Similar to \texttt{beq}, except this instruction skips \texttt{i} instructions if the contents of the two registers are not equal
\item \texttt{slt}: set if less than. Structured as \texttt{slt \$d, \$s, \$t}, which compares register \$s and \$t, and if \$s $<$ \$t, then we set \$d to 1; otherwise we set \$d to 0
\item \texttt{sltu}: similar to \texttt{slt}, except it compares the values in both registers as \textit{unsigned}. Unsigned is another way of saying ``natural numbers" (in CS 241, natural numbers start at 0. This means that unsigned numbers are only positive), whereas signed means ``integers" (which means it includes negative numbers)
\end{itemize}
\subsection{How the PC changes when we branch}
Whenever we branch (using either \texttt{bne} or \texttt{beq}) the PC skips $i$ instructions, and since each instruction is 4 bytes long, we add $i \times 4$ to the current PC. In addition to that, we increment the PC by 4 which happens each time an instruction gets executed. Thus, whenever we branch, the PC changes by \underline{PC = PC + 4 + ($i \times 4$)}
\section{Memory Access in MIPS}
The maximum size of memory in a 32-bit architecture is $2^{32}$ bytes = 4 GB. It's acceptable to think of RAM as one big array, which stores data in a distinct location in memory called an address. To access memory, we can access any of the $2^{32}$ bytes directly; however, it's more convenient to access any of the $2^{30}$ words directly, and that's how MIPS accesses addresses. This is why each address in MIPS is divisible by 4. We access this memory using two instructions:
\begin{itemize}
\item \texttt{lw}: load word. Structured as \texttt{lw \$t, i(\$s)}, which loads a word from memory address (\$s + i)$^*$ into \$t
\item \texttt{sw}: store word. Structured as  \texttt{sw \$t, i(\$s)}, which stores a word from \$t into memory address (\$s + i)$^*$
\end{itemize}
$^*$ = address must be divisible by 4
\section{More Arithmetic Operations in MIPS}
We can also multiply and divide values in MIPS. These operations use special registers \textit{hi} and \textit{lo}:
\begin{itemize}
\item \texttt{mult \$s, \$t}: multiplies the contents of registers \$s and \$t. Since our answer can be larger than 32 bits, we place the most significant 32 bits in \textit{hi} and the least significant 32 bits in \textit{lo} 
\item \texttt{div \$s, \$t}: divides \$s by \$t. Places the result in \textit{lo} and the remainder in \textit{hi} 
\end{itemize}
These instructions also work with unsigned integers; \texttt{multu} and \texttt{divu} (these instructions take in the same arguments as their respective ``signed" counterparts).
\subsubsection{Accessing hi and lo}
To access the hi register, use instruction \texttt{mfhi \$d} to copy the contents into register \$d. \\
To access the lo register, use instruction \texttt{mflo \$d} to copy the contents into register \$d.
%END%
\end{document}