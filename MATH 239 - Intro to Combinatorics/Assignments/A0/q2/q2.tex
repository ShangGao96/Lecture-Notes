\documentclass{report}
\usepackage[margin=1in, paperwidth=8.5in, paperheight=11in]{geometry}
%Math packages%
\usepackage{amsmath}
\usepackage{amsthm}
%Spacing%
\usepackage{setspace}
\onehalfspacing

\begin{document}
\textbf{MATH 239 | Assignment 0, Question 2}
\begin{enumerate}
\item[a)] Since there are two possible values for each digit in a binary string, for a string of length $n$ there are $2^n$ possibilities.

\item[b)] There exist only two binary strings of length 2 where each digit is distinct | ``10" and ``01". Also, there are $2^{n-2}$ possible binary strings of length $n-2$ (using the same logic as shown in part (a)).\\
Combining these two values, we see that that there are $2^{n-2} \times 2 = 2^{n-1}$ binary strings of length $n$ where the first two digits are distinct.

\item[c)] There exist zero binary strings of length $n$ where the first three digits are all distinct simply because it's impossible to have three unique digits in a binary string.
\end{enumerate}
\end{document}