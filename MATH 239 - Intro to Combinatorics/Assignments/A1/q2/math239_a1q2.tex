\documentclass{report}
\usepackage[margin=1in, paperwidth=8.5in, paperheight=11in]{geometry}
%Math packages%
\usepackage{amsmath}
\usepackage{amsthm}
%Spacing%
\usepackage{setspace}

\begin{document}
\noindent \textbf{MATH 239 | Assignment 1, Question 2}\\
Bartosz Antczak (ID: 20603468)
\onehalfspacing
\begin{enumerate}
\item[a)] 
I am mapping the vertices of graph $G$ onto graph $H$\\
\begin{center}
\begin{tabular}{ c | c }
$x$ & $f(x)$ \\ \hline
1 & $P$ \\ \hline
2 & $R$ \\ \hline
3 & $O$ \\ \hline
4 & $D$ \\ \hline
5 & $U$ \\ \hline
6 & $C$ \\ \hline
7 & $T$ \\ \hline
8 & $I$ \\ \hline
9 & $V$ \\ \hline
10 & $E$ \\ \hline
11 & $L$ \\ \hline
12 & $Y$
\end{tabular}
\end{center}
\item[b)] We see that in graph $H$, there exist five vertices of degree 3. For these five vertices, we see that:
\begin{itemize}
\item Vertex $O$ has neighbours with degree \{3, 3, 4\}
\item Vertex $T$ has neighbours with degree \{3, 4, 5\}
\item Vertex $E$ has neighbours with degree \{2, 3, 3\}
\item Vertex $L$ has neighbours with degree \{3, 3, 5\}
\item Vertex $Y$ has neighbours with degree \{2, 3, 4\}
\end{itemize}
Now, in graph $J$, we see that vertex $f$ is also degree 3; however, the respective degrees of its neighbours are \{2, 3, 5\}, which is not found for any third-degree vertex in graph $H$, thus $H$ and $J$ are non-isomorphic.
\end{enumerate}
\end{document}