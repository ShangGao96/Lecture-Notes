\documentclass{report}
\usepackage[margin=1in, paperwidth=8.5in, paperheight=11in]{geometry}
%Math packages%
\usepackage{amsmath}
\usepackage{amsthm}
%Spacing%
\usepackage{setspace}

\begin{document}
\noindent \textbf{MATH 239 | Assignment 1, Question 3}\\
Bartosz Antczak (ID: 20603468)
\onehalfspacing
\begin{enumerate}
\item For any graph with at least two vertices, assume that no two vertices have the same degree. This means that for a graph with $n$ vertices, we can map a distinct degree to each one:
\begin{itemize}
\item Vertex $A_1$ will have a degree of 0
\item Vertex $A_2$ will have a degree of 1\\
$\vdots$
\item Vertex $A_n$ will have a degree of $n-1$
\end{itemize}
However, if vertex $A_n$ has $n-1$ edges, this means that $A_n$ is neighbours with every other vertex, including $A_1$, but that can't happen because $A_1$ has zero edges | a contradiction.\\
Therefore, any graph with at least 2 vertices will contain two vertices of the same degree.
\end{enumerate}
\end{document}