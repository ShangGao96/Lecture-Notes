\documentclass{report}
\usepackage[margin=1in, paperwidth=8.5in, paperheight=11in]{geometry}
%Math packages%
\usepackage{amsmath}
\usepackage{amsthm}
\usepackage{amssymb, mathrsfs}
%Spacing%
\usepackage{setspace}
\onehalfspacing
%Lecture number%
\newcommand{\lectureNum}{5}
%Variables - Date and Course%
\newcommand{\curDate}{January 13, 2017}
\newcommand{\course}{MATH 239}
\newcommand{\instructor}{Peter Nelson (substitute)}
%Defining the example tag%
%\theoremstyle{definition}%
\newtheorem{ex}{Example}[section]
%Setting counter given the lecture number%
\setcounter{chapter}{\lectureNum{}}
%Package for drawing graphs%
\usepackage{tikz}
\usepackage{verbatim}
\usetikzlibrary{arrows}

\begin{document}
%Note title%
\begin{center}
\begin{Large}
\textsc{\course{} | Lecture \lectureNum{}}
\end{Large}
\end{center} 
\noindent \textit{Bartosz Antczak} \hfill
\textit{\curDate{}}
\rule{\textwidth}{0.4pt}

% Actual Notes%
\subsubsection{Note:}
\begin{center}
\textit{I was away for this lecture, so today's notes will be substituted with a summary of key points on the respective topics pertaining to this lecture from the provided course notes online. (I'll cover sections 4.7 and 4.8 from the course notes here)}
\end{center}
\section{Equivalence Relations}
We define a relation $\mathscr{R}$ between two sets $S$ and $T$ as a subset of $S \times T$. For instance, if $a \in S$ and $b \in T$ (where our main set is $S \times T$), then $a$ and $b$ are \textit{related} or \textit{incident}. These relations can have several properties:
\begin{itemize}
\item \textbf{Reflexivity}: if each element in $S$ is related to itself
\item \textbf{Symmetric}: if $a$ is related to $b$, then $b$ is related to $a$ (an example is the relation ``divides" for integers. This relation is reflexive (i.e., 7 divides 7) but is not symmetric (i.e., 3 divides 9 does not imply that 9 divides 3))
\item \textbf{Transitive}: if $a$ is related to$b$ and $b$ is related to $c$, then $a$ is related to $c$
\end{itemize}
We say a relation is an \textbf{equivalence relation} is it satisfies all three of the previous properties.

\section{Connectedness}
A graph $G$ is \textit{connected} if, for each two vertices $u$ and $v$, there is a path from $u$ to $v$.
\subsection{Components}
A \textbf{component} of $G$ is a subgraph $C$ such that
\begin{enumerate}
\item $C$ is connected
\item No subgraph of $G$ that properly contains $C$ is connected
\end{enumerate}
%END%
\end{document}